\documentclass{beamer}
\usepackage[utf8]{inputenc}
\usepackage[frenchb]{babel}
\usepackage{tikz}
\makeatletter\def\@makecaption{}\makeatother
\usepackage[scriptsize]{caption}
\renewcommand*{\figurename}{}
\usetikzlibrary{shapes,positioning,snakes,calc,chains}
\usetheme{Frankfurt}
\usepackage{graphicx}

\title{FMIN327 Cognition individuelle et collective\\ Protocoles artificiels entre agents naturels}
\author{DUPÉRON Georges \and\\ BONAVERO Yoann}
\institute{Université Montpellier II,\\Département informatique  \\ Master 2 IFPRU \\ Sous la direction de Monsieur Jacques Ferber}
\date{Jeudi, 3 novembre 2011}

\defbeamertemplate*{footline}{shadow theme}
{%
  \leavevmode%
  \hbox{\begin{beamercolorbox}[wd=.5\paperwidth,ht=2.5ex,dp=1.125ex,leftskip=.3cm plus1fil,rightskip=.3cm]{author in head/foot}%
    \usebeamerfont{author in head/foot}\insertframenumber\,/\,\inserttotalframenumber%\hfill\url{http://www.pticlic.fr/}
  \end{beamercolorbox}%
  \begin{beamercolorbox}[wd=.5\paperwidth,ht=2.5ex,dp=1.125ex,leftskip=.3cm,rightskip=.3cm plus1fil]{title in head/foot}%
    \usebeamerfont{title in head/foot}\insertshorttitle%
  \end{beamercolorbox}}%
  \vskip0pt%
}

\AtBeginSection[] { 
  \begin{frame}
    \frametitle{Plan} 
    \tableofcontents[currentsection] 
  \end{frame} 
  \addtocounter{framenumber}{-1} 
}

\begin{document}
\renewcommand*{\figurename}{}

\begin{frame}
  \titlepage
\end{frame}

\section{Introduction}

\section{Génération}

\subsection{Perlin noise}
% Ridged Perlin Noise

% [Démo de Ridged Perlin Noise](http://www.inear.se/2010/04/ridged-perlin-noise/)

%     // Fait des crêtes de montagnes ou vallées.
%     abs(perlinNoise());

% Hills Algorithm
% ---------------

% Inverse de craters : on ajoute plein de cercles :

%     repeat 1000 times :
%         r=random();
% 	cx=random();
% 	cy=random();
% 	terrain[x][y] += r**2 + ((x-cx)**2 – (y-cy)**2)

% Craters
% -------

% Soustraire des cercles (profondeur = f(distance au centre)) au terrain
% existant.
% Ou : générer un terrain nu, et soustraire plein de cercles aléatoirs.

% Erosion
% -------

% Modélisation correcte : trop lent. À la place, outil "courbes" de gimp.

% Rivières
% ========

% [Pathfinding pour créer des rivières](http://www.umbrarumregnum.net/articles/creating-rivers).
% Si on utilise une méthode de coût qui favorise de passer par un petit
% bout de bruit plutôt que de le contourner, mais favorise le
% contournement pour une grosse accumulation de bruit, on pourra même
% simuler l'érosion qui efface les méandres trop petits.


\section{Rendu}

\subsection{Isosurfaces}

\subsection{Ray casting}

% Ma démo
\begin{frame}
  \begin{itemize}
  \item Très simple à implémenter
  \item Très bons résultats avec du sampling
  \item Très lent
  \end{itemize}
\end{frame}

\subsubsection{Monte carlo}

\section{Niveau de détail}

\subsection{ROAM}

\subsection{CLOD}

\subsection{Notre algo}

% Triangle fans

\section{Streaming de scène}

\begin{frame}
  \texttt{/usr/lib/xscreensaver/crackberg}
\end{frame}

\section{Conclusion}

\begin{frame}
  \frametitle{Conclusion}
\end{frame}

\begin{frame}
  \frametitle{Sources}
% Génération
% * [Différents algos](http://www.sluniverse.com/php/vb/project-development/34994-automatically-generated-terrain-map.html) : Ridged Perlin Noise, Hills Algorithm, Craters, Erosion.
% * [Plein d'algos](http://planetgenesis.sourceforge.net/docs15/noise/noise.html#tileworley) dont plusieurs basés sur une sorte de voronoi donc à priori trop lents.
% * Affichage avec Ogre : [forum](http://www.ogre3d.org/forums/viewtopic.php?f=5&t=67177&p=442222), [doc](http://www.ogre3d.org/docs/api/html/classOgre_1_1BillboardSet.html)
  \begin{itemize}
  \item gamasutra
  \item vterrain
  \item mojoworld generator
  \item ***world machine***
  \item http://www-cs-students.stanford.edu/~amitp/game-programming/polygon-map-generation/
  \item \dots
  \end{itemize}
\end{frame}

\end{document}
